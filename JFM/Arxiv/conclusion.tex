\section{Conclusion}
\label{conclusion}
A center-manifold reduction for the Navier--Stokes equation is derived with the extension of the system to include the trivial parameter evolution equation. This inclusion however enlarges the critical subspace in a non-trivial manner. The system is then reformulated to bring it to an appropriate form and the application of center-manifold theorem leads to a graph equation for the reformulated system. The graph equation is a representation of the infinite dimensional stable subspace that is driven by the dynamics in the critical subspace. The equation is solved asymptotically and expressions for the asymptotic solutions are obtained, which finally result in a set of equations for center-manifold amplitude equations for the critical eigenvectors. The derivation is essentially agnostic to the dimension of the critical subspace however, the number terms required for the asymptotic evaluation rises rapidly with increasing critical subspace dimension. While the derivation is set in the context of the Navier--Stokes, the methodology is more general and equations~\eqref{general_second_order_expression} and \eqref{general_higher_order_expression} point to the essential structure of the asymptotic solutions of the graph equation that may be obtained for other infinite-dimensional problems. 

The proposed method with system extension is of particular relevance when considering center-manifold reduction for problems that have been perturbed away from the bifurcation point and is a formal way to incorporate parameter perturbations within the asymptotic approximations of the graph equation thereby negating the need for a double asymptotic expansion or an apriori assumption of the normal form of the reduced dynamics.

The derivation is then applied to two cases - the case of the Hopf bifurcation of a cylinder wake and flow in an open cavity resulting in the reduced center-manifold amplitude equations from which, the Stuart-Landau equations are derived. The results from the reduced equations provide a good prediction for the frequencies of the full system close to the bifurcation point. In particular, the linearized eigenvalue variation is predicted up to fourth order in $\eta$ for the case of the cylinder flow and it is found to match well with the spectra calculation of the full system at different Reynolds numbers. The case of flow in an open cavity has interesting dynamics with clear evidence of mode switching at a certain parameter value. This mode switching behavior can not be captured within the bounds of center-manifold theory. Nevertheless, the frequency prior to the mode switching is indeed predicted well by the center-manifold reduction of the extended system.

\section*{Acknowledgements}
The author is grateful for the computational resources and support provided by the Scientific Computing and Data Analysis section of Research Support Division at OIST.

The research was supported by the Okinawa Institute of Science and Technology Graduate	University (OIST) with subsidy funding from the Cabinet Office, Government of Japan. Part of the research was conducted at the Nordic Institute for Theoretical Physics, Nordita, and the author acknowledges the support of Nordita and the Swedish Research Council Grant No. 2018-04290.


\FloatBarrier
